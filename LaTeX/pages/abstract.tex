\chapter{\abstractname}

Microcontrollers are all around us and used in many different devices. They fulfill particular tasks and are subject to many constraints, such as small memory and less processing power. In this thesis, we will look at the feasibility of running WebAssembly on an ESP32 Microcontroller. WebAssembly is a newly developed bytecode meant to serve as a compilation target that can be used in any browser to execute optimized code at near-native speeds. Recently the interest around running WebAssmbly on embedded devices has picked up, and we want to evaluate how WebAssembly can be run on the ESP32 microcontroller. For this, we found the WASM3 runtime that can interpret and execute WebAssembly on the ESP32 and performs better than all other currently known WebAssembly Interpreters. To test the execution, we designed a collection of test workloads inspired by requirements that programs on a Microcontroller might have. We ran them while measuring the execution time of native code compared to WebAssembly code. Our tests show that the execution times increase by up to 90x when interpreting the code as WebAssmbly. While the lower performance and limited support for system interaction pose severe drawbacks to using WebAssembly, there are also significant advantages. New languages that support WebAssembly as a compilation target can be used without being explicitly supported by the platform and modules can be dynamically loaded over the air and executed on the microcontroller without the need for flashing the system.
